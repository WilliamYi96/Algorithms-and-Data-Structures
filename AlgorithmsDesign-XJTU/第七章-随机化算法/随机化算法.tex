\documentclass[UTF8]{ctexart}
\usepackage{amsmath}
\usepackage{graphicx}
\usepackage{float}
\usepackage{subfigure}
\usepackage{xeCJK}
\usepackage{hyperref}
\usepackage{algorithm2e}
\usepackage{amsfonts}
\usepackage{epsfig}
\usepackage{listings}
\usepackage{xcolor}
% 定义可能使用到的颜色

\definecolor{CPPLight}  {HTML} {686868}
\definecolor{CPPSteel}  {HTML} {888888}
\definecolor{CPPDark}   {HTML} {262626}
\definecolor{CPPBlue}   {HTML} {4172A3}
\definecolor{CPPGreen}  {HTML} {487818}
\definecolor{CPPBrown}  {HTML} {A07040}
\definecolor{CPPRed}    {HTML} {AD4D3A}
\definecolor{CPPViolet} {HTML} {7040A0}
\definecolor{CPPGray}  {HTML} {B8B8B8}
\lstset{
    columns=fixed,
    numbers=left,                                        % 在左侧显示行号
    frame=none,                                          % 不显示背景边框
    backgroundcolor=\color[RGB]{245,245,244},            % 设定背景颜色
    keywordstyle=\color[RGB]{40,40,255},                 % 设定关键字颜色
    numberstyle=\footnotesize\color{darkgray},           % 设定行号格式
    commentstyle=\it\color[RGB]{0,96,96},                % 设置代码注释的格式
    stringstyle=\rmfamily\slshape\color[RGB]{128,0,0},   % 设置字符串格式
    showstringspaces=false,                              % 不显示字符串中的空格
    language=c++,                                        % 设置语言
    morekeywords={alignas,continute,friend,register,true,alignof,decltype,goto,
    reinterpret_cast,try,asm,defult,if,return,typedef,auto,delete,inline,short,
    typeid,bool,do,int,signed,typename,break,double,long,sizeof,union,case,
    dynamic_cast,mutable,static,unsigned,catch,else,namespace,static_assert,using,
    char,enum,new,static_cast,virtual,char16_t,char32_t,explict,noexcept,struct,
    void,export,nullptr,switch,volatile,class,extern,operator,template,wchar_t,
    const,false,private,this,while,constexpr,float,protected,thread_local,
    const_cast,for,public,throw,std},
}

\graphicspath{{images/}}
\setCJKmonofont{Microsoft YaHei}

\title{\Huge{计算机算法设计与分析\\ 随机化算法}}
\author{\Huge{易凯}}
\date{\Huge{2017年5月28日}}

\begin{document}
    \maketitle
    \vspace{35mm}
    \begin{flushright}
    \Large{
    \textbf{班\ \ \ \ \ 级} \makebox[5em][l]{软件53班}

    \textbf{学\ \ \ \ \ 号} \makebox[5em][l]{2151601053}

    \textbf{邮\ \ \ \ \ 箱} \makebox[5em][l]{williamyi96@gmail.com}

    \textbf{联系电话} \makebox[5em][l]{13772103675}

    \textbf{个人网站} \makebox[5em][l]{https://williamyi96.github.io}

                      \makebox[5em][l]{williamyi.tech}

      \textbf{实验日期} \makebox[5em][l]{2017年5月28日}

    \textbf{提交日期} \makebox[5em][l]{2017年6月6日}
    }
    \end{flushright}

    \newpage
  	\tableofcontents
  	\newpage
  	\listoffigures
    \newpage

    \section{学习基本目标}
    1. 理解产生伪随机数的算法

    2. 掌握数值概率算法的设计思想

    3. 掌握蒙特卡洛算法的设计思想

    4. 掌握拉斯维加斯算法的设计思想

    5. 掌握舍伍德算法的设计思想

    \section{随机化算法类型}
    随机化算法共有四大类。分别为数值随机化算法,蒙特卡洛算法,拉斯维加斯算法和舍伍德算法。
    \subsection{数字随机算法}
    数值随机算法往往得到的是问题的近似解,其解的精度随计算时间的增加而不断提高。
    \subsection{蒙特卡洛算法}
    求问题的精确解,但是不保证得到的解的正确性。算法所用的时间越多,得到的正确解的概率越高。
    \subsection{拉斯维加斯算法}
    拉斯维加斯算法不会得到不正确的解,但是可能找不到解。其找到正确解的概率随着所用计算时间的增加而提升。
    \subsection{舍伍德算法}
    舍伍德算法总能求得问题的一个解,且所求的解总是正确的。它设法消除的是最坏情形行为与特定实例之间的关联性。
    \section{随机化计算}
    \subsection{题目描述}
    试设计一个随机化算法计算$365!/340!\ 365^{25}$, 并精确到4位有效数字。
    \subsection{题目解答}
    此题是生日问题的实例化,可以使用Stirling公式进行近似:
    
    $n!=\sqrt{2\pi n} (\frac{n}{e}) [1 + \theta(\frac{1}{n})]$
    
    可以得到
    
    $n!/[(n-k)!n^k]\approx e^{-k^2/(2n)}$
    
    进一步求解得到:
    
    $n!/[(n-k)!n^k]\approx e^{-k(k-1)(2n-k^3)/(6n^2)\pm O(max(k^2/n^2), k^4/n^3)}$
    
    最后带入n=365,k=25得到原式的大小为0.4311。
    
    \section{易验证问题的算法}
    \subsection{问题描述}
    一个问题是易验证的是指对该问题的给定实例的每一个解,都可以有效地验证其正确性。例如:求一个整数的非平凡因子问题是易验证的,而求一个整数的最小平凡因子就不是易验证的。在一般情况下,易验证问题未必是易解的。

    1. 给定一个解易验证问题P的蒙特卡洛方法,设计一个相应的解问题P的拉斯维加斯算法;

    2. 给定一个解易验证问题P的拉斯维加斯算法,设计一个相应的界问题P的蒙特卡洛算法。
    \subsection{问题解答}
    如果给定一个解易验证问题P的MC,设计一个相应的解问题P的LV如下(验证其正确性的算法为Prove()):
    
    \begin{small}
    \begin{lstlisting}
    void Lv(LP x) {
        bool l = Mc(x);
        while(!Prove(x,l)) l = Mc(x);
    }
    \end{lstlisting}
    \end{small}
    
    如果给定一个解易验证问题P的LV,设计一个相应的解问题P的MC如下:
    
    \begin{small}
    \begin{lstlisting}
    bool Mc(LP x) {
        Lv(x);
        if(timeused > maxt) return false;
        else return true;
    }
    \end{lstlisting}
    \end{small}
    
    \section{整数因子分解}
    \subsection{题目描述}
    假设已有一个算法Prime(n)可用于测试整数n是否为一个素数。另外还有一个算法Split(n)可以实现对合数n的因子分割。试利用这两个算法设计一个对给定整数n进行因子分解的算法。
    \subsection{问题解答}
    结合两个算法实现的对给定整数n的因子分解如下:
    
    \begin{small}
    \begin{lstlisting}
    void fact(int n) {
        if(Prime(n)) {output(n); return;}
        int i = Split(n);
        if(i > 1) fact(i);
        if(n > i) fact(n/i);
    }
    \end{lstlisting}
    \end{small}
    
    
    \section{算法的正确率}
    \subsection{题目描述}
    设mc(x)是一致的75\% 正确的蒙特卡洛算法,考虑下面的算法:

    \begin{small}
    \begin{lstlisting}
    mc3(x) {
        int t, u, v;
        t = mc(x);
        u = mc(x);
        v = mc(x);
        if((t==u) || (t==v)) return t;
        return v;
    \end{lstlisting}
    \end{small}

    1. 试证明上述算法mc3(x)是一致的27/32正确的算法,因此是84\%正确的。

    2. 试证明如果mc(x)不是一致的,则mc3(x)的正确率可能低于71\%。
    \subsection{问题解答}
    由于MC返回的为bool类型,因此要么其正确,要么不正确。由上述的t==u和t==v的条件分析可以知道,一方面重复三次MC得到的各次正确的分布为000,001,010,011,100,101,110,111。
    
    其中011, 101, 110, 111可以返回正确解,从而返回正确解的概率为:
    
    $0.25*0.75*0.75 + 0.75*0.25*0.75 + 0.75*0.75*0.25 + 0.75*0.75*0.75 = \frac{27}{32}$。
    
    对于第二问,如果MC(x)是非一致的,那么101不能够保证返回正确解,从而返回正确解的概率可能为:
    
    $0.25*0.75*0.75 + 0.75*0.75*0.25 + 0.75*0.75*0.75 = 0.703 < 0.71$,
    
    因而MC(x)不一致,算法的正确率可能低于0.71.
    
    \section{基于蒙特卡洛算法设计拉斯维加斯算法}
    \subsection{题目描述}
    设算法A和B是解统一判定问题的两个有效的蒙特卡洛算法。算法A是p正确偏真算法,算法B是q正确偏假算法。试利用这两个算法设计一个解同一个问题的拉斯维加斯算法,并使所得到的算法对任何实例的成功率尽可能高。
    \subsection{题目解答}
    成功率最高的方法类似于将两者或起来。
    
    \begin{small}
    \begin{lstlisting}
    bool Lv(LP x) {
        while(1) {
            if(A(x)) return true;
            if(!B(x)) return false;
        }
    }
    \end{lstlisting}
    \end{small}


\end{document} 